\documentclass[hyperref=unicode]{beamer}

\usepackage[nil,bidi=basic-r]{babel}
\babelprovide[import=ar-DZ, main]{arabic}
\babelprovide{english}
\babelfont{rm}{Amiri}
\babelfont{sf}{Amiri}



\usepackage{lua-rl-beamer}

% Themes without Navigation Bars
%\mode<presentation>{\usetheme{default}}
%\mode<presentation>{\usetheme{boxes}}
%\mode<presentation>{\usetheme{Bergen}}
%\mode<presentation>{\usetheme{Boadilla}}
%\mode<presentation>{\usetheme{Madrid}}
%\mode<presentation>{\usetheme{AnnArbor}}
%\mode<presentation>{\usetheme{CambridgeUS}}
%\mode<presentation>{\usetheme{EastLansing}}
%\mode<presentation>{\usetheme{Pittsburgh}}
%\mode<presentation>{\usetheme{Rochester}}

% Themes with a Tree-Like Navigation Bar
%\mode<presentation>{\usetheme{Antibes}}
%\mode<presentation>{\usetheme{JuanLesPins}}
%\mode<presentation>{\usetheme{Montpellier}}

% Themes with a Table of Contents Sidebar
%\mode<presentation>{\usetheme{Berkeley}}
%\mode<presentation>{\usetheme{PaloAlto}}
%\mode<presentation>{\usetheme{Goettingen}}
%\mode<presentation>{\usetheme{Hannover}}

% Themes with a Mini Frame Navigation
%\mode<presentation>{\usetheme{Berlin}}
%\mode<presentation>{\usetheme{Ilmenau}}
%\mode<presentation>{\usetheme{Dresden}}
%\mode<presentation>{\usetheme{Darmstadt}}
%\mode<presentation>{\usetheme{Frankfurt}}
%\mode<presentation>{\usetheme{Singapore}}
%\mode<presentation>{\usetheme{Szeged}}

% Themes with Section and Subsection Tables
%\mode<presentation>{\usetheme{Copenhagen}}
%\mode<presentation>{\usetheme{Luebeck}}
\mode<presentation>{\usetheme{Malmoe}}
%\mode<presentation>{\usetheme{Warsaw}}


\title{ميكانيكا}

\author{From Wikipedia}
\date{\today}

\setbeamertemplate{itemize item}[ball]
\setbeamertemplate{itemize subitem}[ball]
\setbeamertemplate{itemize subsubitem}[ball]
\setbeamertemplate{bibliography item}[triangle]
\setbeamercovered{transparent=10}

\begin{document}

\begin{frame}
\titlepage
\end{frame}

\begin{frame}
\frametitle{\contentsname}
\tableofcontents
\end{frame}

\section{تطبيقية}

\begin{frame}
\frametitle{أقسام}
\begin{itemize}
\item فيزياء تطبيقية
\item فيزياء تجريبية
\item فيزياء نظرية
\end{itemize}



\end{frame}

\section{طاقة, حركة}

\begin{frame}
\frametitle{طاقة, حركة}
\begin{itemize}
\item ديناميكا حرارية
\item ميكانيكا
\begin{itemize}
\item كلاسيكية
\begin{itemize}
\item لاغرانج
\item هاملتوني
\end{itemize}
\item المتصل
\item سماوية
\end{itemize}
\end{itemize}
\end{frame}

\begin{frame}
\frametitle{طاقة, حركة}
\begin{enumerate}
\item ديناميكا حرارية
\item ميكانيكا
\begin{enumerate}
\item كلاسيكية
\begin{enumerate}
\item لاغرانج
\item هاملتوني
\end{enumerate}
\item المتصل
\item سماوية
\end{enumerate}
\end{enumerate}

\begin{figure}
\centering
\rule{2cm}{2cm}
\caption{عنوان الصورة}
\end{figure}


\end{frame}


\section{Blocks}

\subsection{text in subsection}
\subsubsection{text in subsubsection}

\begin{frame}
\frametitle{Blocks}


\begin{block}{Lorem}
  \foreignlanguage*{nil}{On 21 April 1820, during a lecture, Ørsted
  noticed a compass\cite{Dijkstra1982} needle deflected from magnetic north when an
  electric current from a battery was switched on and off.}
\end{block}


\begin{block}{أورستد}
  لاحظ هانز أورستد في 21 أبريل 1820 وهو يُعد أحد التجارب أن إبرة
  البوصلة تنحرف عن اتجاهها نحو الشمال عندما كان يغلق ويفتح التيار في
  دائرة كهربائية يُعدها.
\end{block}
\end{frame}

\begin{frame}
\frametitle{columns}
\begin{columns}[t]
\begin{column}{5cm}
نص عربي طويل من اليمين لليسار، مكتوب على سطرين.
\end{column}
\begin{column}{5cm}
\babelsublr{One line (but aligned).}
\end{column}
\end{columns}

\bigskip

\begin{columns}[b]
\begin{column}{5cm}
نص عربي طويل من اليمين لليسار، مكتوب على سطرين.
\end{column}
\begin{column}{5cm}
\babelsublr{One line (but aligned).}
\end{column}
\end{columns}
\end{frame}

\begin{frame}
\frametitle{Theorems}

\framesubtitle{The proof uses \textit{reductio ad absurdum}.}
\begin{theorem}
There is no largest prime number.
\end{theorem}
\begin{proof}
\begin{enumerate}[<+-| alert@+>]
\item Suppose $p$ were the largest prime number.
\item Let $q$ be the product of the first $p$ numbers.
\item Then $q+1$ is not divisible by any of them.
\item But $q + 1$ is greater than $1$, thus divisible by some prime
number not in the first $p$ numbers.\qedhere
\end{enumerate}
\end{proof}

\end{frame}



\begin{frame}[fragile]

\frametitle{Verbatim text.}

\selectlanguage{english}

\begin{verbatim}
int main (void)
{
std::vector<bool> is_prime (100, true);
for (int i = 2; i < 100; i++)
if (is_prime[i])
{
std::cout << i << " ";
for (int j = i; j < 100; is_prime [j] = false, j+=i);
}
return 0;
}
\end{verbatim}
\begin{uncoverenv}<2>
Note the use of \verb|std::|.
\end{uncoverenv}

\end{frame}

\begin{frame}
\frametitle{For Further Reading}
\begin{thebibliography}{Dijkstra, 1982}
\bibitem[Salomaa, 1973]{Salomaa1973}
A.~Salomaa.
\newblock {\em Formal Languages}.
\newblock Academic Press, 1973.
\bibitem[Dijkstra, 1982]{Dijkstra1982}
E.~Dijkstra.
\newblock Smoothsort, an alternative for sorting in situ.
\newblock {\em Science of Computer Programming}, 1(3):223--233, 1982.
\end{thebibliography}
\end{frame}


\end{document}
